\documentclass[a4paper,11pt]{article}
\usepackage{indentfirst}
\usepackage[T1]{fontenc}
\usepackage[polish]{babel}
\usepackage[utf8]{inputenc}
\usepackage{lmodern}
\selectlanguage{polish}
\usepackage[top=2cm, bottom=2cm, left=1cm, right=1cm]{geometry}
\usepackage{lastpage}
\usepackage{fancyhdr}
\pagestyle{fancy}
\setlength\parindent{24pt}
\makeatletter
\newcommand{\linia}{\rule{\linewidth}{0.4mm}}
\renewcommand{\maketitle}{\begin{titlepage}
    \vspace*{2cm}
    \begin{center}\LARGE
    Politechnika Warszawska\\
    Wydział Elektryczny\\
    \end{center}
    \vspace{5cm}
    \noindent\linia
    \begin{center}
      \LARGE \textsc{\@title}
         \end{center}
     \linia
    \vspace{0.5cm}
    \begin{flushright}
    \begin{minipage}{5cm}
    \textit{Autor:}\\
    \normalsize \textsc{\@author} \par
    \end{minipage}
    \vspace{5cm}
     \end{flushright}
    \vspace*{\stretch{6}}
    \begin{center}
    \@date
    \end{center}
  \end{titlepage}
}
\makeatother
\author{Grzegorz Kopyt}
\title{Sprawozdanie \\
,,Arbitrage''}
\usepackage{graphicx}

\fancyhf{}
\rfoot{\thepage{}/\pageref{LastPage}}

\begin{document}
\maketitle

\tableofcontents
\vspace{1cm}
\noindent\linia
\section{Cel powstania dokumentu}
Dokument ma na celu podsumowanie pracy nad projektem ,,Arbitrage". Przedstawia problem, jaki miał zostać rozwiązany, opisuje zastosowany algorytm oraz wskazuje zmiany dokonane w stosunku do wcześniejszych specyfikacji. Dokument zawiera również wnioski z podjętych decyzji oraz refleksje nad zastosowanymi rozwiązaniami.

\noindent\linia
\section{Opis problemu}
Program został postawiony przed dwoma zadaniami:
\begin{itemize}
\item znajdowaniem najkorzystniejszej ścieżki wymiany waluty,
\item znajdowaniem dowolnego arbitrażu.
\end{itemize}

Jako źródło danych otrzymywał specjalnie spreparowany plik, który zawierał definicje walut oraz możliwych wymian (z informacją  o kursach i opłatach).
Na podstawie tego pliku oraz kwoty wejściowej program miał znajdować dowolny arbitraż. Natomiast do znalezienia najkorzystniejszej ścieżki wymiany waluty otrzymywał oprócz pliku i kwoty wejściowej, także walutę wejściową oraz docelową.

\noindent\linia
\section{Wysoko abstrakcyjny opis działania algorytmu}
\begin{enumerate}
\item Algorytm pobiera dane od użytkownika (w zależności od zadania są to dane opisane w sekcji 2).
\item Algorytm przechodzi kilkukrotnie po wszystkich walutach i możliwych wymianach dokonując obliczeń na kwocie wejściowej.
\item W każdej walucie pozostawia największą kwotę oraz ścieżkę wymiany, którą można uzyskać z dotarcia do tej waluty od waluty początkowej.
\item W zależności od zadania:
\begin{enumerate}
\item Wymiana
\begin{enumerate}
\item Algorytm pobiera z waluty docelowej ścieżkę oraz kwotę końcową.
\end{enumerate}
\item Arbitrage
\begin{enumerate}
\item Algorytm pobiera z waluty początkowej, która jest także walutą końcową, ścieżkę wymiany oraz kwotę końcową.
\end{enumerate}
\end{enumerate}
\item Algorytm wyświetla użytkownikowi wyniki swojej pracy.
\end{enumerate}
\noindent\linia
\section{ Efekty działania programu}
\noindent\linia
\section{Zmiany względem specyfikacji}

\noindent\linia
\section{Podsumowanie i wnioski}

\noindent\linia

\end{document}



